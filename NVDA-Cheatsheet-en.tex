\documentclass[a4paper, landscape, 11pt]{scrartcl} 
\usepackage[utf8]{inputenc} 
\usepackage[danish]{babel} 
\usepackage[T1]{fontenc}
\usepackage{multicol}
\usepackage[landscape]{geometry}
\usepackage{textcomp}
\usepackage{fontawesome}
\usepackage{palatino}
\usepackage{amssymb}
\usepackage[os=win]{menukeys}
\renewmenumacro{\keys}[+]{shadowedroundedkeys}
\usepackage{hyperref}
\hypersetup{
  hidelinks
}

\renewcommand{\dots}{\ \dotfill{}\ } % Fills in the right amount of dots
\newcommand{\command}[2]{#1~\dotfill{}~#2\\} % Custom command for adding a shorcut

\begin{document}

\section*{NVDA (Non-Visual Desktop Access Cheatsheet (version-$\beta$) - ole@omgwtf.dk}

\hrulefill{}

\begin{multicols}{2}

\subsection*{NVDA-keys}
NVDA uses the NVDA-key to control with the screenreader. \\ 
\command{NVDA-key (NVDA)}{\keys{Capslock} and (or) \keys{Insert}} \\
This document describes NVDA's Desktop-layout which is default. If you wish you can use a Laptop-layout, but the keystrokes wil be different.

\subsection*{Starting and stopping NVDA}
\command{Start NVDA}{\keys{\ctrl + \Alt + n}}
\command{Quit NVDA}{\keys{NVDA + q}}
\command{Stop speech}{\keys{\ctrl}}
\command{Pause and restart speech}{\keys{\shift}}

\subsection*{NVDA Help}
\command{Start and stop NVDA Input-help}{\keys{NVDA + 1}}
\command{Open NVDA-menu}{\keys{NVDA + n}}

\subsection*{NVDA Settings}
\command{General NVDA-settings}{\keys{NVDA + \ctrl + g}}
\command{NVDA keyboard-settings}{\keys{NVDA + \ctrl + k}}

\subsection*{Windows Settings}
\command{Open Settings}{\keys{\faWindows + i}}
\command{Open Accessibility Settings}{\keys{\faWindows + u}}

\subsection*{Adjust Voice-settings}
\command{Find Voice-setting}{\keys{NVDA + \ctrl + \arrowkeyleft} or \keys{\arrowkeyright}}
\command{Change Voice-setting}{\keys{NVDA + \ctrl + \arrowkeyup} or \keys{\arrowkeydown}}
\command{Open Synthiser-settings}{\keys{NVDA + \ctrl + s}}
\command{Open Voice-settings}{\keys{NVDA + \ctrl + v}}

\subsection*{Reading commands}
\command{Read line}{\keys{NVDA + \arrowkeyup}}
\command{Read element from current position}{\keys{NVDA + \arrowkeydown}}
\command{Read focus}{\keys{NVDA + \tab}}
\command{Read window-title}{\keys{NVDA + t}}
\command{Read window}{\keys{NVDA + b}}
\command{Read text-attributes}{\keys{NVDA + f}}

\subsection*{Object navigation}
\command{Move to next element}{\keys{NVDA + \arrowkeyright}}
\command{Move to previous element}{\keys{NVDA + \arrowkeyleft}}
\command{Interact with element}{\keys{NVDA + \arrowkeydown}}
\command{Stop interaction with element}{\keys{NVDA  + \arrowkeyup}}
\command{Activate element}{\keys{NVDA + \SPACE}}

\subsection*{Navigating the user-interface}
\command{Goto Notifications}{\keys{\faWindows + b}}
\command{Open Activities}{\keys{\faWindows + a}}
\command{Goto Desktop}{\keys{\faWindows + d}}
\command{Open File explorer}{\keys{\faWindows + e}}

\subsection*{Using Navigation-mode}
\command{Start and stop navigationmode}{\keys{NVDA + \SPACE}}
\command{Jump to next heading}{\keys{h}}
\command{Jump to previous heading}{\keys{\shift + h}}
\command{Jump to next link}{\keys{l}}
\command{Jump to previous link}{\keys{\shift + h}}
\command{Jump to next button}{\keys{b}}
\command{Jump to previous button}{\keys{\shift + b}}
\command{Jump to next editfield}{\keys{e}}
\command{Jump to previous editfield}{\keys{\shift + e}}
\command{Jump to next graphic}{\keys{g}}
\command{Jump to previous graphic}{\keys{\shift + g}}
\command{Describe graphic}{\keys{Na + \ctrl + d}}

\subsection*{Gennemsyn af tekst}
You can review text without moving the textcaret - use the numeric keys. \\ \\
\command{Previous, current, next line}{\keys{7}, \keys{8}, \keys{9}}
\command{Previous, current, next word}{\keys{4}, \keys{5}, \keys{6}}
\command{Previous, current, next character}{\keys{1}, \keys{2}, \keys{3}}

\subsection*{Lists}
To open a list of the different  elements that you can navigate. \\ \\
\command{Open elementlist}{\keys{NVDA + F7}}

\end{multicols}

\hrulefill{}

\end{document}
