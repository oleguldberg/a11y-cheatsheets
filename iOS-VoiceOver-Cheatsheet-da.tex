\documentclass[a4paper, landscape, 10pt]{scrartcl} 
\usepackage[utf8]{inputenc} 
\usepackage[danish]{babel} 
\usepackage[T1]{fontenc}
\usepackage{multicol}
\usepackage[landscape]{geometry}
\usepackage{textcomp}
\usepackage{palatino}
\usepackage{amssymb}
\usepackage{fontawesome}
\usepackage[os=win]{menukeys}
\renewmenumacro{\keys}[+]{shadowedroundedkeys}
\usepackage{hyperref}
\hypersetup{
  hidelinks
}

\begin{document}
\thispagestyle{empty}

\section*{iOS Tilgængelighed Snydeark (version-\faUniversalAccess) - Ole Guldberg}

\hrulefill{}
\begin{multicols}{2}

\subsection*{Indstille VoiceOver}
Åben App'en \textbf{\textit{Indstillinger}} for at indstille tilgængelighed.
\begin{itemize}
  \item {Tilgængelighedsgenvej: \menu[,]{Tilgængelighed,Tilgængelighedsgenvej}.}
  \item {VoiceOver: \menu[,]{Tilgængelighed,VoiceOver}.}
\end{itemize}

\subsection*{Starting and stopping VoiceOver}
\begin{itemize}
  \item{Start VoiceOver med \emph{Siri}: "Start VoiceOver ".}
  \item{Stop VoiceOver med \emph{Siri}: "Stop VoiceOver".}
  \item {Start (og stop) med 3 tryk på \emph{tilgængelighedsgenvej}.}
\end{itemize}

\subsection*{Oplæsning med VoiceOver}
\begin{itemize}
  \item{Svirp $\upuparrows$ for at læse fra begyndelsen.}
  \item{Svirp $\downdownarrows$for at læse fra nuværende position.}
  \item{Brug ($\cdot\cdot$) pause og genstart oplæsning.}
\end{itemize}

\subsection*{Basisnavigation med VoiceOver}
\begin{itemize}
  \item{Svirp $\rightarrow$ for næste element.}
  \item{Svirp $\leftarrow$ for forrige element.}
  \item{Brug ($\cdot\cdot$) for at aktivere element.}
  \item{Bevæg ($\cdot\cdot\cdot$) op eller ned for at scrolle.}
  \item{Tap ($\cdot\cdot\cdot\cdot$) for første element.}
  \item{Tap ($\cdot\cdot\cdot\cdot$) for sidste element.}
  \item{Zigzag ($\cdot\cdot$) for at gå tilbage.}
\end{itemize}

\subsection*{Andre VoiceOver-bevægelser}
\begin{itemize}
  \item{Tap ($\cdot\cdot\cdot$) for information hvor du er.}
  \item{Tap ($\cdot\cdot\cdot$) ($\cdot\cdot\cdot$) for at stoppe og starte tale.}
  \item{Tap ($\cdot\cdot\cdot$)($\cdot\cdot\cdot$)($\cdot\cdot\cdot$) tænde og slukke skærmen.}
\end{itemize}

\subsection*{VoiceOver - Magisk tap}
Tap ($\cdot\cdot$) ($\cdot\cdot$) er kendt som  \textbf{\textit{magisk tap}} og er kontekst afhængigt.
\begin{itemize}
  \item{Tap ($\cdot\cdot$)($\cdot\cdot$) start og stop diktering.}
  \item{Tap ($\cdot\cdot$)($\cdot\cdot$) start og stop telefonsamtale.}
  \item{Tap ($\cdot\cdot$)($\cdot\cdot$) start og stop afspilling af musik.}
\end{itemize}

\subsection*{Brug af VoiceOver Rotor-funktionen}
Rotoren er et kontekst-afhængigt værktøj. 
\begin{itemize}
  \item{Brug rotor-bevægelsen (to-fingere) $\circlearrowleft{}$ eller $\circlearrowright$ til at indstille rotoren-funktionen.}
  \item{Svirp $\downarrow$ for at flytte til forrige rotor-element.}
  \item{Svirp $\uparrow$ for at flytte til næste rotor-element.}
\end{itemize}

\newpage

\subsection*{Indstille tilgængelighed}
Åben App'en \textbf{\textit{Indstillinger}} for at indstille tilgængelighed. \\ \\
Vær opmærksom på indstillingerne for \emph{Zoomområder} og \emph{Zoomfilter} i under indstillingerne for \textbf{\emph{Zoom}}.
\begin{itemize}
  \item {Tilgængelighedsgenvej: \menu[,]{Tilgængelighed,Tilgængelighedsgenvej}.}
  \item {Zoom: \menu[,]{Tilgængelighed,Zoom}.}
  \item {Forstør: \menu[,]{Tilgængelighed,Forstørrelsesglas}.}
  \item {Oplæst indhold: \menu[,]{Tilgængelighed,Oplæst indhold}.}
\end{itemize}

\subsection*{Forstørrelse - Zoom}
\begin{itemize}
  \item{Tap ($\cdot\cdot\cdot$)($\cdot\cdot\cdot$) for at start og stoppe forstørrelse.}
  \item{Tap ($\cdot\cdot\cdot$)($\cdot\cdot\cdot$) \emph{og} bevæg fingrene op eller ned for sætte forstørrelsesgrad}
\end{itemize}

\subsection*{Oplæst indhold}
\begin{itemize}
  \item{Svirp ($\cdot\cdot$) fra toppen af skærmen og ned for at læse skærmens indhold op.}
  \item{Læs skærm på med \emph{Siri}: \emph{Læs skærm op}.}
  \item{Vælg elementer og vælg \emph{Læs op}.}
\end{itemize}

\subsection*{\emph{Siri} - den digitale assistent}
\begin{itemize}
  \item {Indstil \emph{Siri}: \menu[,]{Indstillinger,Siri \& søgning}.}
  \item {Hvad kan \emph{Siri}?: \emph{Hvad kan du?} }
\end{itemize}

\subsection*{Diktering af tekst}
\begin{itemize}
  \item {Indstil diktering: \menu[,]{Indstillinger,Generelt,Tastatur}.}
  \item {Start diktering fra tastaturet: \faMicrophone.}
\end{itemize}

\end{multicols}
\hrulefill{}
\end{document}
